% Nathan J. Goldbaum - Curriculum Vitae
%
% This is based on a template, http://jblevins.org/projects/cv-template/
% original copyright notice below.
%
% Copyright (C) 2004-2016 Jason R. Blevins <jrblevin@sdf.org>
% http://jblevins.org/
%
% You may use use this document as a template to create your own CV
% and you may redistribute the source code freely.  No attribution is
% required in any resulting documents.  I do ask that you please leave
% this notice and the above URL in the source code if you choose to
% redistribute this file.

\documentclass[10pt,letterpaper]{article}

\usepackage{hyperref}
\usepackage{geometry}
\usepackage{enumitem}
\usepackage[compact]{titlesec}
\usepackage{mdwlist}
\usepackage{tabularx}
\usepackage{fancyhdr}

% Fonts
\usepackage[T1]{fontenc}
\usepackage{lmodern}

% Set your name here
\def\name{Nathan J. Goldbaum}

% The following metadata will show up in the PDF properties
\hypersetup{
  colorlinks = true,
  urlcolor = black,
  pdfauthor = {\name},
  pdfkeywords = {astrophysics, open source, visualization, python},
  pdftitle = {\name: Curriculum Vitae},
  pdfsubject = {Curriculum Vitae},
  pdfpagemode = UseNone
}

\geometry{
  body={7.0in, 9.0in},
  left=0.75in,
  top=1.0in
}

% Customize page headers
\pagestyle{myheadings}
\markright{\name}
\thispagestyle{empty}

% Custom section fonts
\usepackage{sectsty}

\sectionfont{\rmfamily\bfseries\Large}
\subsectionfont{\rmfamily\bfseries\itshape\large}

% Don't indent paragraphs.
\setlength\parindent{0em}

\newcolumntype{Y}{>{\raggedleft\arraybackslash}X}

\newcommand{\textline}[2]{
  \begin{tabularx}{\textwidth}{XY}
  #1 & #2
  \end{tabularx}
}

\renewcommand*{\thefootnote}{\fnsymbol{footnote}}
\setcounter{footnote}{1}

\renewcommand{\labelitemii}{$\bullet$}

\setlist[1]{leftmargin=0.5em, itemsep=0.4em, parsep=0em, topsep=0.25em, partopsep=0pt}
\setlist[2]{leftmargin=3em, itemsep=0.5em, parsep=0em, topsep=0.25em, partopsep=0pt}


\fancyhf{}
\renewcommand{\headrulewidth}{0pt}
\fancyfoot[R]{Last updated: \today}
\pagestyle{fancy}

\begin{document}

% Place name at left
{\huge \name}

% Alternatively, print name centered and bold:
%\centerline{\huge \bf \name}

\bigskip

\begin{minipage}[t]{0.6\textwidth}
  National Center for Supercomputing Applications \\
  University of Illinois Urbana-Champaign \\
  1205 W Clark St \\
  Urbana, IL 61801
\end{minipage}
\begin{minipage}[t]{0.4\textwidth}
  Phone: (720) 201-2231 \\
  Office: 4028 NCSA \\
  Email: \href{mailto:ngoldbau@illinois.edu}{ngoldbau@illinois.edu} \\
  Homepage: \href{https://github.com/ngoldbaum}{https://github.com/ngoldbaum}
\end{minipage}

\section*{Employment}

\subsection*{National Center for Supercomputing Applications}
\begin{itemize}
  \item[] \textline{Research Scientist}{2017 -- Present}
  \item[] \textline{Postdoctoral Researcher, Data Exploration Lab}{2015 -- 2017}
\end{itemize}

\section*{Education}

\subsection*{University of California Santa Cruz}
\emph{Dissertation:} ``Star Formation in Graivtationally Unstable Disk
Galaxies: From Clouds to Disks''\\
\emph{Committee:} M.~R. Krumholz, J.~X. Prochaska, J. Primack
\begin{itemize}
\item[] \textline{Ph.D. Astronomy \& Astrophysics}{2011 -- 2015}
\item[] \textline{NSF Graduate Research Fellow}{2010 -- 2013}
\item[] \textline{M.S. Astronomy \& Astrophysics}{2009 -- 2011}
\end{itemize}

\subsection*{University of Colorado Boulder}
\begin{itemize}
  \item[] \textline{B.A. Physics, \textit{Summa Cum Laude}}{2005 -- 2009}
\end{itemize}

\section*{Funded Grants}

\subsection*{NSF SI2-SSI 2016}
Co-Principal Investigator on NSF award \#1663914, \textit{Inquiry-Focused
  Volumetric Data Analysis Across Scientific Domains: Sustaining and Expanding
  the yt Community}, totaling \$1.6 Million to UIUC, Columbia University, and
University of Wisconsin. This grant will expand the usage of yt to fields
beyond theoretical astrophysics, including meteorology, seismology, geophysics,
oceanography, observational astronomy, and plasma physics simulations.

\section*{Open Source Contributions\footnote{See
    \href{https://www.openhub.net/accounts/ngoldbaum}{https://www.openhub.net/accounts/ngoldbaum}
    for a contribution summary}}

\subsection*{yt}
An analysis and visualization toolkit for volumetric data.

\begin{itemize}

\item[] \textline{\bf Core Contributor, Steering Committee Member, and Project Member}{2012 -- Present}
  \begin{itemize}
    \item Regularly contributes to project management and planning, including
      design and implementation discussions, code review, and user support on
      IRC and mailing lists.
    \item More than 650 merged pull requests
  \end{itemize}

\item[] \textline{\bf Improving support for particle data}{2016 -- Present}
  \begin{itemize}
  \item Redesigned high-level \texttt{yt} API for analyzing and visualizing data
    produced by smoothed particle hydrodynamics simulations.
  \item Substantially improved performance and memory usage for common analysis
    tasks.
  \item Worked with community to provide migration path for analysis results
    requiring data produced by old API.\@
  \end{itemize}

\item[] \textline{\bf Symbolic Unit System}{2013 -- 2014}
  \begin{itemize}
    \item Designed \texttt{YTArray}, an array data container for automatically
      handling unit conversions and runtime validation for dimensional correctness
      of mathematical operations.
    \item Led the development effort that systematically updated \texttt{yt} to
      make use of the unit system.
  \end{itemize}

\item[] \textline{\bf \texttt{PlotWindow} Plotting Interface}{2012}
  \begin{itemize}
    \item Created an interface for visualizing slices and projections of
      simulation data.
    \item Enables quick data visualization through an API that focuses on what
      the simulation data physically represents rather than how it is laid out
      on disk.
    \item Used in more than a dozen published journal articles written by
      \texttt{yt} users.
    \item Responsible for the initial design, implementation, documentation, and maintenance.
  \end{itemize}

\end{itemize}

\subsection*{Enzo}

A freely available MPI-parallel hydrodynamics and N-body simulation code with a
focus on high-order hydrodynamics and a suite of physics packages intended for
simulations of galaxy formation.

\begin{itemize}
\item[] \textline{Contributor}{2012 -- Present}
  \begin{itemize}
  \item Authored more than 50 merged pull requests
  \item Regularly participates in developer discussions and user support on mailing lists and IRC.
  \end{itemize}
\item[] \textline{Isolated Galaxy Initial Conditions}{2014}
  \begin{itemize}
  \item Created an initial conditions generator for a simulation of a disk
    galaxy similar to the Milky Way
  \item Open sources a class of disk galaxy models that spent a decade locked in
    proprietary closed-source codes.
  \end{itemize}
\item[] \textline{Active Particles}{2012 -- 2015}
  \begin{itemize}
  \item Collaborated on designing and implementing a new infrastructure for
    particle types that can dynamically alter the simulation as it runs.
  \end{itemize}
\end{itemize}

\subsection*{Additional Contributions}

\begin{itemize}
\item[] {\bf The Grackle}: Added an automated test suite for continuous
  integration
\item[] {\bf xonsh}: Added integration with the mercurial version control System
\item[] {\bf Mercurial}: Several bugfixes and minor improvements
\item[] {\bf Jupyter}: Provided bugfixes and minor improvement to several
  Jupyter projects including \texttt{ipywidgets}, \texttt{IPython}, and
  \texttt{nbconvert}.
\end{itemize}

\section*{Publications}

\begin{itemize}
  \item[] ``GRACKLE: a chemistry and cooling library for astrophysics``, Smith,
    B.~D., Bryan, G.~L., Glover, S.~C.~O., {\bf Goldbaum, N.~J.}, and 10
    co-authors, {\it Monthly Notices of the Royal Astronomical Society}, 2017,
    466, 2217
  \item[] ``The AGORA High-resolution Galaxy Simulations Comparison
    Project. II. Isolated Disk Test``, Kim, J.~H.\ and 43 co-authors, including
    {\bf Goldbaum, N.~J.}, {\it Astrophysical Journal}, 2016, 833, 202
  \item[] ``Mass Transport and Turbulence in Gravitationally Unstable Disk Galaxies. II:
    The Effects of Star Formation Feedback'', {\bf Goldbaum, N.~J.}, Krumholz, M.~R., \& Forbes,
    J.~C., {\it Astrophysical Journal}, 2016, 827, 28
  \item[] ``Suppression of Star Formation in Dwarf Galaxies by Photoelectric
    Grain Heating Feedback'', Forbes, J.~C., Krumholz, M.~R., {\bf Goldbaum,
      N.~J.}, Dekel, A., {\it Nature}, 2016, 535, 523
  \item[] ``Mass Transport and Turbulence in Gravitationally Unstable Disk Galaxies. I:
    The Case of Pure Self-Gravity'', {\bf Goldbaum, N.~J.}, Krumholz, M.~R., \& Forbes,
    J.~C., {\it Astrophysical Journal}, 2015, 814, 131
  \item [] ``Mixing and Transport of Metals by Gravitational Instability-Driven
    Turbulence'' Petit, A.~C., Krumholz,
    M.~R., {\bf Goldbaum, N.~J.}, Forbes, J.~C., {\it Monthly Notices of
      the Royal Astronomical Society}, 2015, 449, 2588
  \item[] ``\texttt{Enzo} An Adaptive Mesh Refinement Code for Astrophysics'', The
    Enzo Collaboration and 19 co-authors including {\bf Goldbaum, N.~J.}, {\it
      Astrophysical Journal Supplement}, 2014, 211, 19
  \item[] ``The AGORA High-Resolution Galaxy Simulations Comparison Project'',
    Kim, J.~H.\ and 45 co-authors, including {\bf Goldbaum, N.~J.}, {\it
      Astrophysical Journal Supplement}, 2014, 210, 14
  \item[] ``Dwarf Galaxies with Ionizing Radiation Feedback. II. Spatially
    Resolved Star Formation Relation'', Kim, J.~H., Krumholz, M.~R., Wise,
    J.~H., Turk, M.~J., {\bf Goldbaum, N.~J.}, Abel, T., {\it Astrophysical Journal},
    2013, 779, 8
  \item[] ``The Global Evolution of Giant Molecular Clouds II: The Role of
    Accretion'', {\bf Goldbaum, N.~J.}, Krumholz, M.~R., Matzner, C.~D., McKee,
    C.~F.\ {\it Astrophysical Journal}, 2011, 738, 101
  \item[] ``The Intensity Profile of the Solar Supergranulation'', {\bf Goldbaum,
      N.~J.}, Rast, M.~P., Ermolli, I., Sans, J.~S., Berilli, F.\ {\it
      Astrophysical Journal}, 2009, 707, 67

\end{itemize}

\section*{Talks and Proceedings}

\begin{itemize}
\item[] ``The Demeshening: The Next Generation of Particle Support in yt'',
  {\bf Goldbaum, N.~J.}, Lang, M.~M., SciPy, 2017
\item[] ``Domain Specific Visualization'', {\bf Goldbaum, N.~J.}, PlotCon NYC,
  2016
\item[] ``Mass Transport and ISM Phase Balance in Simulations of Gravitationally
  Unstable Disk Galaxies'', {\bf Goldbaum, N.~J.}, Great Lakes Cosmology and
  Galaxies, 2016
\item[] ``Publicly Releasing a Large Simulation Dataset'', {\bf Goldbaum,
    N.~J.}, Python in Astronomy Conference, 2016
\item[] ``\texttt{yt}: Volumetric Data Analysis'', {\bf Goldbaum, N.~J.} on
  behalf of the \texttt{yt} project, SciPy conference, 2014
\item[] ``Using \texttt{yt} With \texttt{Athena} Data'', {\bf Goldbaum, N.~J.},
  HIPACC Galaxy formation summer school, 2013
\item[] ``From Gas to Stars: Simulating a Population of GMCs'', {\bf Goldbaum,
    N.~J.}, Krumholz M.~R., 2013, \textit{Proc. IAUS 292}, 91
\item[] ``The Role of Accretion in the Evolution of Giant Molecular Clouds'',
  {\bf Goldbaum, N.~J.}, Krumholz M.~R., 2010, \textit{From Stars to Galaxies},
  J. Tan \& S. Van Loo (eds.)

\end{itemize}

\end{document}
