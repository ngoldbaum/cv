% Nathan J. Goldbaum - Curriculum Vitae
%
% This is based on a template, http://jblevins.org/projects/cv-template/
% original copyright notice below.
%
% Copyright (C) 2004-2016 Jason R. Blevins <jrblevin@sdf.org>
% http://jblevins.org/
%
% You may use use this document as a template to create your own CV
% and you may redistribute the source code freely.  No attribution is
% required in any resulting documents.  I do ask that you please leave
% this notice and the above URL in the source code if you choose to
% redistribute this file.

\documentclass[10pt,letterpaper]{article}

\usepackage{hyperref}
\usepackage{geometry}
\usepackage{enumitem}
\usepackage[compact]{titlesec}
\usepackage{mdwlist}
\usepackage{tabularx}
\usepackage{fancyhdr}

% Fonts
\usepackage[T1]{fontenc}
\usepackage{lmodern}

% Set your name here
\def\name{Nathan J. Goldbaum}


\geometry{
  body={7.0in, 10in},
  left=0.75in,
  top=0.5in
}

% Customize page headers
\pagestyle{myheadings}
\markright{\name}
\thispagestyle{empty}

% Custom section fonts
\usepackage{sectsty}

\sectionfont{\rmfamily\bfseries\Large}
\subsectionfont{\rmfamily\bfseries\itshape\large}  %chktex 6

% Don't indent paragraphs.
\setlength\parindent{0em}

\newcolumntype{Y}{>{\raggedleft\arraybackslash}X}

\newcommand{\textline}[2]{
  \begin{tabularx}{\textwidth}{XY}
  #1 & #2
  \end{tabularx}
}

\renewcommand*{\thefootnote}{\fnsymbol{footnote}}
\setcounter{footnote}{1}

\renewcommand{\labelitemii}{$\bullet$}

\setlist[1]{leftmargin=0.5em, itemsep=0.4em, parsep=0em, topsep=0.25em, partopsep=0pt}
\setlist[2]{leftmargin=3em, itemsep=0.5em, parsep=0em, topsep=0.25em, partopsep=0pt}


\fancyhf{}
\renewcommand{\headrulewidth}{0pt}
\fancyfoot[R]{Last updated: \today}
\pagestyle{fancy}

\begin{document}

% Place name at left
{\large \bf \name}

% Alternatively, print name centered and bold:
%\centerline{\huge \bf \name}

\bigskip

\begin{minipage}[t]{0.6\textwidth}
  2770 JT Coffman Dr \\
  Champaign, IL 61822
\end{minipage}
\begin{minipage}[t]{0.4\textwidth}
  Phone: (720) 201-2231 \\  %chktex 8
  Email: \href{mailto:nathan.goldbaum@gmail.com}{nathan.goldbaum@gmail.com} \\
  Homepage: \href{ngoldbaum.github.io}{ngoldbaum.github.io}
\end{minipage}

\section*{Experience}

\subsection*{Recurse Center}
\begin{itemize}
  \item [] \textline{Participant}{May 2019 -- Present}
    \begin{itemize}
      \item [] Transitioned from academia to industry by attending a
        self-directed educational retreat. Explored concepts in algorithms, data
        structures, and theoretical computer science by implementing a client
        for the Mercurial version control system in Rust. Wrote blog posts about
        concepts in distributed version control systems, binary data formats
        used by Mercurial, the Rust programming languages, and general
        programming topics.
    \end{itemize}    
\end{itemize}

\subsection*{National Center for Supercomputing Applications}
\begin{itemize}
  \item[] \textline{Postdoctoral Researcher}{August 2015 -- April 2017}  \textline{Research
    Scientist}{May 2017 -- April 2019} %chktex 8
  \begin{itemize}
    \item [] Primary maintainer for The \texttt{yt} Project, an open source toolkit
      for analysis and visualization of 3D simulation data in Python. Regularly
      contributed to project management and planning, including design and
      implementation discussions, code review, and supporting users with a range
      of technical expertise via live chat, mailing lists, and in-person
      workshops. More than 650 merged pull requests for \texttt{yt} alone. Many
      other merged pull requests in popular open source projects such as
      \texttt{Matplotlib}, \texttt{Mercurial}, \texttt{xonsh},
      \texttt{conda-forge}, \texttt{Homebrew}, \texttt{h5py}, \texttt{IPython},
      \texttt{ipywidgets}, and \texttt{nbconvert}.
  \end{itemize}

\end{itemize}

\section*{Projects}

\begin{itemize}

  \item[] {\bf \texttt{unyt}}
    \begin{itemize}
      \item [] A Python library for working with data that has physical
        units. Reworked \texttt{yt.units} module into a standalone \texttt{unyt}
        package with continuous integration, 100\% test coverage, and extensive
        example-driven documentation with entry points for users doing quick
        calculations and users who want to extensively use \texttt{unyt} in an
        application. Achieved substantially improved performance while
        simultaneously shipping user-requested features like automatic unit
        simplification and automatic unit name canonicalization.
    \end{itemize}
  \item[] {\bf Improved support for particle data in \texttt{yt}}
    \begin{itemize}
      \item [] Improved performance and memory usage for common analysis tasks by
        10x to 100x by leveraging a novel system for spatially indexing particle
        data via EWAH-compressed Morton codes. Enabled production science
        pipelines leveraging gigabyte and terabyte scale simulation
        outputs. Worked with community to provide migration path for analysis
        results requiring data produced by old API.\@ Presented work to the
        community at a SciPy conference talk.
    \end{itemize}
  \item[] {\bf\texttt{PlotWindow} Plotting Interface}
    \begin{itemize}
      \item [] Created an interface for visualizing slices and projections in
        \texttt{yt}. Enables quick data visualization through an API that
        focuses on what the simulation data physically represent rather than how
        they are laid out on disk. Used in dozens of published journal articles
        written by \texttt{yt} users.
    \end{itemize}

\end{itemize}


\section*{Education}

\subsection*{University of California Santa Cruz}
\emph{Dissertation:} ``Star Formation in Gravitationally Unstable Disk
Galaxies: From Clouds to Disks''\\
\begin{itemize}
\item[] \textline{Ph.D. Astronomy \& Astrophysics}{August 2011 -- July 2015}  %chktex 8
%\item[] \textline{NSF Graduate Research Fellow}{August 2010 -- August 2013}  %chktex 8
\item[] \textline{M.S. Astronomy \& Astrophysics}{August 2009 -- July 2011}  %chktex 8
\end{itemize}

\subsection*{University of Colorado Boulder}
\begin{itemize}
  \item[] \textline{B.A. Physics, \textit{Summa Cum Laude}}{August 2005 -- 2009}  %chktex 8
\end{itemize}

\end{document}
