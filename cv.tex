% Nathan J. Goldbaum - Curriculum Vitae
%
% This is based on a template, http://jblevins.org/projects/cv-template/
% original copyright notice below.
%
% Copyright (C) 2004-2016 Jason R. Blevins <jrblevin@sdf.org>
% http://jblevins.org/
%
% You may use use this document as a template to create your own CV
% and you may redistribute the source code freely.  No attribution is
% required in any resulting documents.  I do ask that you please leave
% this notice and the above URL in the source code if you choose to
% redistribute this file.

\documentclass[10pt,letterpaper]{article}

\usepackage{hyperref}
\usepackage{geometry}
\usepackage{enumitem}
\usepackage[compact]{titlesec}
\usepackage{mdwlist}
\usepackage{tabularx}
\usepackage{fancyhdr}

% Fonts
\usepackage[T1]{fontenc}
\usepackage{lmodern}

% Set your name here
\def\name{Nathan J. Goldbaum}


\geometry{
  body={7.0in, 10in},
  left=0.8in,
  right=0.8in,
  top=0.35in
}

% Customize page headers
\pagestyle{myheadings}
\markright{\name}
\thispagestyle{empty}

% Custom section fonts
\usepackage{sectsty}

\sectionfont{\rmfamily\bfseries\large}
\subsectionfont{\rmfamily\bfseries\itshape\small}  %chktex 6

% Don't indent paragraphs.
\setlength\parindent{0em}

\newcolumntype{Y}{>{\raggedleft\arraybackslash}X}

\newcommand{\textline}[2]{
  \begin{tabularx}{\textwidth}{XY}
  #1 & #2
  \end{tabularx}
}

\renewcommand*{\thefootnote}{\fnsymbol{footnote}}
\setcounter{footnote}{1}

\renewcommand{\labelitemii}{$\bullet$}

\setlist[1]{leftmargin=0.5em, itemsep=0.4em, parsep=0em, topsep=0.25em, partopsep=0pt}
\setlist[2]{leftmargin=3em, itemsep=0.5em, parsep=0em, topsep=0.25em, partopsep=0pt}


\fancyhf{}
\renewcommand{\headrulewidth}{0pt}
\fancyfoot[R]{Last updated: \today}
\pagestyle{fancy}

\begin{document}

% Place name at left
{\huge \bf \name}

% Alternatively, print name centered and bold:
%\centerline{\huge \bf \name}

\bigskip

\begin{minipage}[t]{0.4\textwidth}
  Phone: (720) 201-2231 \\  %chktex 8
  Email: \href{mailto:nathan.goldbaum@gmail.com}{nathan.goldbaum@gmail.com} \\
  Homepage: \href{ngoldbaum.github.io}{ngoldbaum.github.io}
\end{minipage}

\section*{Major Open Source Contributions}

\begin{itemize}

  \item[] \textline{\texttt{NumPy}}{\href{https://numpy.org}{https://numpy.org}}
    \begin{itemize}
      \item Foundational library for multidimensional arrays in Python.
      \item Joined maintainer team in 2023.
      \item Author of the variable-width string array data type added in
        \texttt{NumPy} 2.0 and described in
        \href{https://numpy.org/neps/nep-0055-string_dtype.html}{NEP 55}.
      \item Led effort to support free-threaded Python in \texttt{NumPy} 2.1 and
        primary maintainer for the free-threaded build.
      \item More than 200 merged pull requests. Thirteenth most prolific all-time contributor by commit count.
    \end{itemize}

  \item[] \textline{\texttt{PyO3}}{\href{https://pyo3.rs}{https://pyo3.rs}}
    \begin{itemize}
      \item The Rust bindings for the Python interpreter.
      \item Joined maintainer team in 2024.
      \item Led effort to support free-threaded Python in \texttt{PyO3} 0.23 and
        primary maintainer for the free-threaded build.
      \item More than 40 merged pull requests.
    \end{itemize}

  \item[] \textline{\bf \texttt{PyTorch}}{\href{https://pytorch.org}{https://pytorch.org}}
    \begin{itemize}
      \item Contributed the \texttt{\_\_torch\_function\_\_} mechanism for
        extending PyTorch in downstream libraries and many bugfixes.
      \item 20 merged pull requests.
    \end{itemize}

    
  \item[] \textline{\bf \texttt{unyt}}{\href{https://github.com/yt-project/unyt/}{\small https://github.com/yt-project/unyt/}}
    \begin{itemize}
      \item A Python library for working with data that has physical
        units.
      \item Primary author and original maintainer.
      \item Extensively documented and tested Python codebase that can turns a
        Python prompt into a full-featured calculator for problems with physical
        units.
      \item Used in production for scientific and engineering data processing
        pipelines.
    \end{itemize}

\end{itemize}

\section*{Experience}
\begin{itemize}
  \item [] \textline{{\bf Quansight Labs}, Senior Software Engineer}{November 2022 - Present}
    \begin{itemize}
      \item Improving and maintaining the scientific Python ecosystem.
    \end{itemize}
  \item [] \textline{{\bf Quansight}, Software Engineer}{October 2019 -- March 2020}
    \begin{itemize}
      \item Worked with \texttt{PyTorch} team to handle community-reported bugs
        and feature requests.
    \end{itemize}
  \item [] \textline{{\bf Recurse Center}}{May 2019 -- August 2019}
    \begin{itemize}
    \item Gained facility in the \texttt{Rust} programming language by blogging
      about self-directed learning experiences.
    \end{itemize}    
    
  \item [] {\bf National Center for Supercomputing Applications}
  \item[] \textline{Research Scientist}{May 2017 -- April 2019} % chktex 8
    \textline{Postdoctoral Researcher}{August 2015 -- April 2017} % chktex 8
    \begin{itemize}
    \item Primary maintainer for The \texttt{yt} Project, an open toolkit
      for working with 3D simulation data.
    \end{itemize}
\end{itemize}

\section*{Education}

\subsection*{University of California Santa Cruz}
\begin{itemize}
\item[] \textline{Ph.D. Astronomy \& Astrophysics}{August 2011 -- July 2015}  %chktex 8
\end{itemize}

\subsection*{University of Colorado Boulder}
\begin{itemize}
  \item[] \textline{B.A. Physics, \textit{Summa Cum Laude}}{August 2005 -- 2009}  %chktex 8
\end{itemize}

\end{document}
